\documentclass[11pt]{beamer}
\usepackage{amsmath,amssymb,amsthm}
\usepackage{algorithm}
\usepackage[noend]{algpseudocode} 
\usepackage{titlesec}
\usepackage{enumitem}
\usepackage[T2A,T1]{fontenc}
\usepackage{amsmath,amssymb,amsthm}
\usepackage{fancyhdr}
\usepackage{indentfirst}

%---enable russian----

\usepackage[utf8]{inputenc}
\usepackage[russian]{babel}

% PROBABILITY SYMBOLS
\newcommand*\PROB\Pr 
\DeclareMathOperator*{\EXPECT}{\mathbb{E}}


% Sets, Rngs, ets 
\newcommand{\N}{{{\mathbb N}}}
\newcommand{\Z}{{{\mathbb Z}}}
\newcommand{\R}{{{\mathbb R}}}
\newcommand{\Zp}{\ints_p} % Integers modulo p
\newcommand{\Zq}{\ints_q} % Integers modulo q
\newcommand{\Zn}{\ints_N} % Integers modulo N

% Landau 
\newcommand{\bigO}{\mathcal{O}}
\newcommand*{\OLandau}{\bigO}
\newcommand*{\WLandau}{\Omega}
\newcommand*{\xOLandau}{\widetilde{\OLandau}}
\newcommand*{\xWLandau}{\widetilde{\WLandau}}
\newcommand*{\TLandau}{\Theta}
\newcommand*{\xTLandau}{\widetilde{\TLandau}}
\newcommand{\smallo}{o} %technically, an omicron
\newcommand{\softO}{\widetilde{\bigO}}
\newcommand{\wLandau}{\omega}
\newcommand{\negl}{\mathrm{negl}} 

% Misc
\newcommand{\eps}{\varepsilon}
\newcommand{\inprod}[1]{\left\langle #1 \right\rangle}

\usetheme{Darmstadt}
\usecolortheme{wolverine}

\title{Курсовая работа
\\
{\small по дисциплине: «Основы информационной безопасности»}}
\subtitle{Тема: «Средства резервного копирования и восстановления данных: принципы работы, виды, обзор российского и мирового рынка»}
\author{Гудасов Александр, 2 курс КБ}
\institute{Балтийский федеральный университет имени Иммануила Канта}

\begin{document}
	
	\begin{frame}
		\titlepage
	\end{frame}
		\begin{frame}
		\begin{block}{Задачи курсовой работы}
			\begin{itemize}
				\item Анализ принципов работы и видов резервного копирования;
				\item Обзор российского и мирового рынка средств резервного копирования и восстановления данных;
				\item Выявление проблем, с которыми может столкнуться пользователь средства резервного копирования и восстановления данных. 
				\end{itemize}
		\end{block}
		\begin{block}{Методы решения}
				Подробное рассмотрение каждой поставленной задачи.
						\end{block}
	\end{frame}
	\begin{frame}
Результаты:
		\begin{itemize}	
				\item Изучены методы резервного копирования;
				\item Исследованы классификации резервного копирования;
				\item Рассмотрены частые проблемы резервного копиирования;
				\item Проанализированы мировой и российский рынок.
			\end{itemize} 

\end{frame}	
\end{document}



