% !TeX spellcheck = ru_RU
\documentclass[twoside,12pt]{article}
\usepackage{titlesec}
\usepackage{enumitem}
\usepackage[T2A,T1]{fontenc}
\usepackage[utf8]{inputenc}
\usepackage[russian]{babel}
\usepackage{amsmath,amssymb,amsthm}
\usepackage{fancyhdr}
\pagestyle{fancy}
\fancyhf{}
\fancyfoot{}
\renewcommand{\headrulewidth}{0pt}
\setlength{\headheight}{15pt}
\usepackage[a4paper,bindingoffset=0.2in,%
left=1in,right=1in,top=1in,bottom=1in,%
footskip=.10in]{geometry}
%table
\newcommand{\eps}{\varepsilon}
\newcommand{\inprod}[1]{\left\langle #1 \right\rangle}


\newcommand{\handout}[5]{
	\noindent
	\begin{center}
		\framebox{
			\vbox{
				\hbox to 5.78in { {\bf Научно-исследовательская практика} \hfill #2 }
				\vspace{4mm}
				\hbox to 5.78in { {\Large \hfill #5  \hfill} }
				\vspace{2mm}
				\hbox to 5.78in { {\em #3 \hfill #4} }
			}
		}
	\end{center}
	\vspace*{4mm}
}

\newcommand{\lecture}[4]{\handout{#1}{#2}{#3}{#4}{Cистема верстки LaTeX  #1}}





\begin{document}
\lecture{}{Лето 2020}{}{Гудасов Александр}	
\newpage	
	
\renewcommand{\headrulewidth}{0pt}
\lhead{\textbf{\thepage}}
\setcounter{page}{160}
\chead{\textbf{Примитивные корни и индексы}}
\rhead{\textbf{Глава 8}}	
	
	

{\large Одно из главных достоинств примитивных корней приведено в следующей теореме.
	
\bigskip

\makeatletter \newenvironment{subtheorem}[1]{%
	\def\subtheoremcounter{#1}%
	\refstepcounter{#1}%
	\protected@edef\theparentnumber{\csname the#1\endcsname}%
		\setcounter{#1}{7}
	\setcounter{parentnumber}{\value{#1}}%
	\setcounter{#1}{3}%
	\expandafter\def\csname the#1\endcsname{8.\arabic{#1}}%
	\ignorespaces
}{%
	\setcounter{\subtheoremcounter}{\value{parentnumber}}%
	\ignorespacesafterend
}
\makeatother
\newcounter{parentnumber}


\newtheorem{thm}{\textsc{Теорема}}
 
\begin{subtheorem}{thm}\label{thm:two}
	\begin{thm}\label{eightfour:8-4}
	\textit{Пусть $\text{НОД}(a,n)=1$, а элементы посследовательности  $a_{1}, a_{2},...,a_{\phi(n)}$ будут положительными целыми числами меньше $n$, также являющимися взаимно простыми с $n$. Если a примитивный корень $n$, тогда 
		$$a^{1}, a^{2},...,a^{\phi(n)}$$ будут сравнимы по модулю с $a_{1}, a_{2},...,a_{\phi(n)}$ в некотором порядке.}	
	\end{thm}
\end{subtheorem}

 

\begin{proof}
Так как $a$ и $n$ взаимно просты, то степени $a$ взаимно просты $n$. Из этого следует, что каждое $a^{k}$ делится на $a_{i}$ по модулю $n$.
Числа $\phi(n)$ в последовательности $\{a^{1}, a^{2},...,a^{\phi(n)}\}$ неконгруентны по следствию из теоремы $8-2$, значит эти степени должны составлять (не обязательно в написанном порядке) числа $a_{1}, a_{2},...,a_{\phi(n)}$.
\end{proof}
	
 \bigskip
Из доказанной теоремы можно сделать следующий вывод: в тех случаях, когда существует примитивный корень, можно сказать сколько всего существует примитивных корней.

\bigskip
\newtheorem{corollary}{\textsc{Следствие}}
\begin{corollary}
\textit{Если у $n$ существует примитивный корень, тогда существует ровно $\phi(\phi(n))$ примитивных корней.} 
\end{corollary}

\begin{proof} Предположим, что $a$ - примитивный корень $n$. По теореме любой примитивный корень находится из последовательности $a^{1}, a^{2},...,a^{\phi(n)}$. Но количество степеней $a_{k}$, $1\leq k \leq\phi(n)$, где порядок $\phi(n)$ равен количеству целых чисел $k$, для которых  $\text{НОД}(k,\phi(n)$)=1. Существует $\phi(\phi(n))$ таких целых чисел, то есть $\phi(\phi(n))$ - количество примитивных корней $n$.
\end{proof}

\bigskip
Теорему \ref{eightfour:8-4} можно проиллюстрировать, взяв $a=2$ и $n=9$. Так как $\phi(9)=6$, то первые шесть степеней $2$ должны делиться по модулю $9$, в некотором порядке, на положительные целые числа меньшие $9$, также степени $2$ должны быть взаимно простыми с $9$. Целые числа, которые меньше и взаимно просты $9$ - это $1,2,4,5,7,8$. \\Видно, что 
$2^{1}\equiv2,2^{2}\equiv4, 2^{3}\equiv8, 2^{4}\equiv7, 2^{5}\equiv5, 2^{6}\equiv1 \pmod{9}$  


\medskip
В силу следствия существует ровно $\phi(\phi(9))=\phi(6)=2$ примитивных корней $9$, это целые числа $2$ и $5$.\newpage
}



\rhead{\textbf{\thepage}}
\chead{\textbf{Порядок целого числа по модулю n}}
\lhead{\textbf{Секция 8-1}}
\begin{center}\textbf{ЗАДАЧИ СЕКЦИИ 8.1}\end{center}
{\footnotesize \begin{enumerate}[label=\textbf{\arabic*.}]
	\item Найдите порядок у целых чисел $2,3,5$: \begin{enumerate}
	   \item по модулю $17$
	   \item по модулю $19$
	   \item по модулю $23$ \end{enumerate}
	\item Проверьте на правильность следующие высказывания:\begin{enumerate}
		\item Если у $a$ существует порядок hk по модулю $n$, тогда у $a^{h}$ есть порядок $k$ по модулю $n$.
		\item Если у $a$ существует порядок $2k$ по модулю нечётного простого числа $p$, тогда $a^{k}\equiv-1 \pmod{p}.$
		\item Если у $a$ существует порядок $n-1$ по модулю $n$, тогда $n$ - простое число. \end{enumerate}
	\item Докажите, что $\phi(2^{n}-1)$ является кратным $n$, для любого $n>1$.
 [\textit{Подсказка:} У целого числа $2$ найдется порядок $n$ по модулю $2^{n}-1$.]
        \item Пусть $h$ - это порядок $a$ по модулю $n, k$ - порядок $b$ по модулю $n$. Докажите, что порядок $ab$ по модулю $n$ делит $hk$; в конкретном случае, где если $\text{НОД}(h,k)=1$, тогда $ab$ будет являться порядком $hk$.
        \item Зная, что порядок $a$ по модулю $p$ равен $3$, где $p$ - нечётное простое число, докажите, что порядком $a+1$ по модулю $p$ будет $6$.
         [\textit{Подсказка:} Зная, что $a^{2}+a+1\equiv0\pmod{p}]$ , можно сделать вывод, что $(a+1)^{2}\equiv a\pmod{p}$.]  
    \item Проверьте на правильность следующие высказывания:\begin{enumerate}
         \item Нечётные простые делители целого числа $n^{2}+1$ имеют вид\break $4k+1$. 
         [\textit{Подсказка:} Так как $n^{2}\equiv-1 \pmod{p}$, где $p$ - нечётное простое число, из теоремы $8-1$ можно сделать вывод, что  $4\mid\phi(p)$.]
         \item Нечётные простые делители целого числа $n^{4}+1$ имеют вид $8k+1$.
         \item Нечётные простые делители целого числа $n^{2}+n+1$, отличные от $3$ имеют вид $6k+1$.	\end{enumerate}
     \item Докажите, что существует бесконечное количество нечётных простых чисел вида $4k+1$, $6k+1$ и $8k+1$. [\textit{Подсказка:} Предположим, что существует конечное число простых чисел вида $4k+1$. Назовём их $p_{1},p_{2},...,p_{r}$. Используйте целое число вида $(2p_{1}p_{2}...p_{r})^{2}+1$ и выполните задание, опираясь на предыдущую задачу.]
     \item \begin{enumerate} \item Докажите, что если $p$ и $q$ нечётные простые числа и $q\mid a^{p}-1$, то выполняется либо $q\mid a-1$ или $q=2kp+1$ для некоторого целого $k$. [\textit{Подсказка:} Так как $a^{p}\equiv1 \pmod{p}$, то порядок $a$ по модулю $q$ будет $1$ или $p$. Во втором случае $p\mid\phi(q)$.]
                              \item Используйте пункт (a), чтобы доказать, что если $p$ - нечётное простое число, то простые делители $2^{p}-1$ примут вид $2kp+1$.
                              \item Найдите наименьший простой делитель для слудующих целых чисел: $2^{17}-1$, $2^{29}-1$.
                              \end{enumerate}
      \item Докажите, что существует бесконечное множество простых чисел вида $2kp+1$, где $p$ - нечётное простое число.[\textit{Подсказка:} Предположим, что существует конечное число простых чисел вида $2kp+1$. Назовём их $q_{1},q_{2},...,q_{r}$. Для решение задания рассмотрите целое число вида $(q_{1}q_{2}...q_{r})^{p}-1$.]     
      \item  \begin{enumerate} \item Проверьте, что $2$ примитивный корень $19$, но не $17$.
      	                       \item Докажите, что у $15$ нет примитивных корней, путём высчитывания порядков $2,4,7,8,11,13,14$ по модулю $15$. \end{enumerate}    
      \item Пусть $r$ - примитивный корень целого числа $n$. Докажите, что $r^{k}$ будет являться примитивным корнем $n$ тогда и только тогда, когда $\text{НОД}(k,\phi(n))=1$.
      \item  \begin{enumerate} \item Найдите $2$ примитивных корня числа $10$.
                               \item Используя информацию, что $3$ - примитивный корень $17$, найдите $8$ примитивных корней $17$.\end{enumerate}
      
	

\end{enumerate}
}\newpage
\rhead{\textbf{\thepage}}
\chead{\textbf{Примитивные корни для простых чисел.}}
\lhead{\textbf{8-2}}
\titleformat*{\subsection}{\Large\bfseries}
\setcounter{section}{8}
\setcounter{subsection}{1}
\subsection{{\Large \textsc{ПРИМИТИВНЫЕ КОРНИ ДЛЯ ПРОСТЫХ ЧИСЕЛ}}}
{\large Примитивнные корни играют важную роль во многих теоретических исследованиях, главная задача исследования примитивных корней сводится к нахождению всех целых чисел, для которых можно найти примитивные корни. На следующих нескольких страницах будет приведено доказательство сущетвования примитивных корней для всех простых чисел. Но перед этим, мы рассмотрим теорему о количестве полиномиальных сравнений.  
\\
\makeatletter \newenvironment{subtheorem}[1]{%
	\def\subtheoremcounter{#1}%
	\refstepcounter{#1}%
	\protected@edef\theparentnumber{\csname the#1\endcsname}%
	\setcounter{#1}{7}
	\setcounter{parenttnumber}{\value{#1}}%
	\setcounter{#1}{4}%
	\expandafter\def\csname the#1\endcsname{8.\arabic{#1}}%
	\ignorespaces
}{%
	\setcounter{\subtheoremcounter}{\value{parenttnumber}}%
	\ignorespacesafterend
}
\makeatother
\newcounter{parenttnumber}


\newtheorem{them}{\textsc{\textsc{Теорема}}}
\begin{subtheorem}{them}\label{thm:three}
	\begin{them}\label{eightfive:8-5}
 \textsc{(Лагранжа)} \textit{Если $p$ простое число и $$f(x)=a_{n}x^{n}+a_{n-1}x^{n-1}+...+a_{1}x+a_{0},\;\;\;\;\;a_{0}\not\equiv0\pmod{p}$$
многочлен степени $n\geqslant1$ с интегральными коэффициентами, тогда у сравнения $$f(x)\equiv0\pmod{p}$$ будет максимум $n$ неконгруентных решений по модулю $p$.}
\end{them}
\end{subtheorem}

\begin{proof}
 Начнём доказательство по методу индукции. Рассмотрим $n$, степень $f(x)$. Если $n=1$, тогда многочлен принимает вид $$f(x)=a_{1}x+a_{0}.$$ Так как $\text{НОД}(a_{1},p)=1$, то по теореме $4.7$, делаем вывод, что сравнение $a_{1}x\equiv-a_{0}\pmod{p}$ имеет уникальное решение по модулю $p$. Поэтому теорема выполняется для $n=1$. 

Теперь по методу индукции предположим, что теорема выполняется для многочленов степени $k-1$ и рассмотрим случай, в котором $f(x)$ имеет степень $k$. В этом случае либо $f(x)\equiv0 \pmod{p}$ не имеет решений (на чём доказательство заканчивается), либо существует одно решение, назовём его $a$. Если многочлен $f(x)$ делится на $x-a$, то результатом будет $$f(x)=(x-a)q(x)+r,$$
\newpage
где $q(x)$ является многочленом степени $k-1$ с интегральными коэффициентами, $r$ - целым числом. Заменяя $x$ на $a$, получим
$$0\equiv f(a)=(a-a)q(a)+r=r\pmod{p}$$
Таким образом $f(x)\equiv(x-a)q(x)\pmod{p}$ 

Если $b$ является одним из неконгруентных решений
\\
 $f(x)\equiv0\;(mod\;p)$, тогда $$0\equiv f(b)=(b-a)q(b)\pmod{p}.$$
Так как $b-a\not\equiv0\pmod{p},$ то $q(b)\equiv0\pmod{p}$; другими словами, любое решение $f(x)\equiv0\pmod{p}$, которое отличается от $a$, должно удовлетворять $q(x)\equiv0\pmod{p}$. По нашему индуктивному предположению, второе сравнение может иметь максимум $k-1$ неконгруентных решений, таким образом $f(x)\equiv0\pmod{p}$ не может иметь более, чем $k$ неконгруентных решений. Этот шаг завершает математическую индукцию и доказательство.
\end{proof}

Из этой теоремы легко выводится следующее следствие:
\\\\

\begin{corollary}
\textit{Если $p$ - простое число и $d\mid p-1$, тогда сравнение $$x^{d}-1\equiv0\pmod{p}$$ имеет  ровно $d$ решений.}
\end{corollary}
\begin{proof}
Так как $d\mid p-1$, получаем $p-1=dk$ для некоторого $k$. Тогда $$x^{p-1}-1=(x^{d}-1)f(x),$$
где многочлен $f(x)=x^{d(k-1)}+x^{d(k-2)}+...+x^{d}+1$ с интегральными коэффициентами в степени $d(k-1)=p-1-d.$	По теореме Лагранжа, сравнение $f(x)\equiv0\pmod{p})$ имеет максимум $p-1-d$ решений. Также по теореме Ферма $x^{p-1}-1\equiv0\pmod{p}$ имеет ровно $p-1$ неконгруентных решений вида $1,2,...,p-1.$ 

Теперь любое решение $x=a$ сравнения $x^{p-1}-1\equiv0 \pmod{p},$ которое не является решением $f(x)\equiv0\pmod{p}$ должно удовлетворять $x^{d}-1\equiv0\pmod{p}$. Для $$0\equiv a^{p-1}-1=(a^{d}-1)f(a)\pmod{p}$$, 
где $p\nmid f(a)$, то есть $p\mid a^{d}-1.$ Из этого следует, что сравнение $x^{d}-1\equiv0\pmod{p}$ должно иметь хотя бы $p-1-(p-1-d)=d$

\end{proof}
\end{document}